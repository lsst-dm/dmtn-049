\documentclass[DM,lsstdraft,toc]{lsstdoc}
\usepackage{graphicx}
\usepackage{url}
\usepackage{color}

\title[Photo-$z$ for LSST Objects]{A Roadmap to Photometric Redshifts for the LSST {\tt Object} Catalog}

\author{M.~L.~Graham, J.~Bosch, L.~P.~Guy, and the DM System Science Team.}

\setDocRef{DMTN-049}
\date{\today}
\setDocRevision{TBD} 
\setDocStatus{draft}

\setDocAbstract{
This roadmap guides the Rubin Observatory Data Management (DM) team towards a decision on the photometric redshift (photo-$z$) estimator(s) that will be implemented into the Data Release (DR) processing pipeline and served for the DR {\tt Object} catalog.
This is a \textit{\textbf{living document}} that will evolve over time to incorporate input from the science community (including the Science Collaborations), the Rubin Observatory DM and Operations teams, and potential in-kind contributors.
This roadmap proposes two phases: (1) soliciting community input to define the criteria for evaluating photo-$z$ estimator(s) via a virtual open forum series in early 2021, and (2) soliciting community input on which photo-$z$ estimator(s) DM should implement via an open call for short science white papers due early 2022. 
{\bf The endpoint of this roadmap is when the DM team chooses one or more community-vetted estimator(s) to be implemented into the DR pipeline.}
}

\setDocChangeRecord{%
\addtohist{1}{2017-04-01}{Initial release of preliminary investigation.}{Melissa Graham}
\addtohist{2}{2018-10-16}{Edited to align with recent DPDD updates, some of which were based on the recommendations of Version 1 of this document.}{Melissa Graham}
\addtohist{3}{2020-XX-XX}{Updated as per ticket/DM-6367.}{Melissa Graham}
}

\begin{document}

\maketitle

% CITATION EXAMPLES
%\verb|\citellp|: \citellp{LPM-17, LSE-30} \\
%\verb|\citell|: (SRD; \citell{LPM-17,LSE-29}) \\
%\verb|\citep[][]|: \citep[e.g.,][are interesting]{LPM-17,LSE-29} \\
%\verb|\cite|: \cite{LPM-17,LSE-29} \\
%\citet{2018A&C....25...58G} \\     % Gschwend et al. [13]
%\citep{2018A&C....25...58G} \\    % [13]
%\citealt{2018A&C....25...58G} \\   % Gschwend et al. 13



% % % % % % % % % % % % % % % % % % % % % % % % % % % % % %
\section{Introduction} \label{sec:intro}

A {\it photometric redshift} is an estimate of an object's cosmological redshift which is based on its photometry (e.g., apparent magnitudes in multiple filters) instead of on its spectral features (e.g., emission and absorption lines). 
Redshift is a key component of many science goals that will be pursued with data from the Legacy Survey of Space and Time (LSST).
Since it will be impossible to obtain spectra for the billions of galaxies that LSST will observe, photometric redshift estimates will be necessary.

Typically, photometric redshift estimators either fit template spectra to the observed photometry or match photometry to a training set of galaxies with spectroscopic redshifts. 
The latter is often done with machine learning codes, and hybrid photo-$z$ estimators also exist. 
Some photo-$z$ estimators are more appropriate for some science goals than others, due to the quality or type of results they produce (e.g., point estimates, full posterior probability density functions, or redshift distributions in tomographic bins).
For this reason, several research groups in the science community are already planning to generate multiple kinds of photo-$z$ (e.g., the Dark Energy Science Collaboration).

It would be scientifically prohibitive if all users of LSST data had to generate their own photo-$z$ estimates, as this is a computational intensive calculation.
Furthermore, it is a requirement\dmreq{0046} that the LSST Data Management System (DMS) calculate photo-$z$ and store them in the {\tt Object} catalog.
Research and development of new photometric redshift algorithms for LSST data is beyond the scope of DM (i.e., not part of DM's specialized knowledge base), as it is an active area of current and future LSST research.
The high-level plan is that one or more existing photo-$z$ estimator(s) be installed into the Data Release (DR) pipeline and run at scale, and/or a user-generated photo-$z$ catalog be ingested and federated with the {\tt Object} catalog.

This roadmap guides the Rubin Observatory DM team towards a decision on which \textit{community-vetted} photo-$z$ estimator(s) will be implemented into the DR pipeline and served for the {\tt Object} catalog.
This is a \textit{\textbf{living document}} that will evolve over time to incorporate input from the science community, which has a considerable wealth of expertise in generating photo-$z$ catalogs and will be the primary users of the photo-$z$ data product.


\clearpage
% % % % % % % % % % % % % % % % % % % % % % % % % % % % % %
\section{Proposed Roadmap and Timeline}\label{sec:time}

All items in this timeline reference {\it actions taken by the Data Management team}. 
Subsection headings propose end dates for each step of the process -- proposed start dates and intermediate milestones are listed in each subsection in chronological order.

Below, the term ``advertise broadly" means posting to Community.lsst.org, sending emails to Science Collaborations and other lists of potentially interested individuals (e.g., authors of papers containing the terms LSST and photo-$z$), and including in Rubin newsletters.
The term ``community" or ``science community" refers to any individuals or groups who plan to use Rubin data products or services for science (and in the particular case of this document, use the {\tt Object} catalog photo-$z$). 
A total of nine virtual, informal, one-hour ``face-to-face" LSST Photo-$z$ Forums are proposed to occur to facilitate community-project interaction and the ingestion of community input. 
It is not intended that all interested people attend all forums -- rather, the many forums planned will be held at a variety of times so as to enable participation from all timezones in at least a couple of forums.
Interaction and feedback will also be facilitated via the Community.lsst.org online forum. 

\subsection{Nov 30 2020: Finalize Roadmap and Timeline}\label{ssec:time_time}

{\bf Sep 30 2020:} Merge the branch {\tt tickets/DM-6367} of this document to master. Write a solicitation for feedback on the roadmap and timeline, and advertise it broadly. \\
{\bf Oct \& Nov 2020:} Hold two virtual, informal, one-hour ``face-to-face" LSST Photo-$z$ Forums that focus on gathering project and community input on the proposed roadmap and its timeline. \\
{\bf Nov 30 2020:} Synthesize feedback and finalize this timeline by updating this document.

\subsection{May 31 2021: Define the Minimum Attributes and Evaluation Criteria}\label{ssec:time_mvp}

{\bf Dec 1 2020:} Write a solicitation for feedback on the minimum attributes (\S~\ref{sec:mvp}) and evaluation criteria (\S~\ref{sec:eval}) and advertise it broadly. \\
{\bf Jan, Mar, \& May 2021:} Hold three virtual, informal, one-hour ``face-to-face" LSST Photo-$z$ Forums that focus on gathering project and community feedback on the proposed minimum attributes and evaluation criteria. \\
{\bf May 31 2021:} Synthesize feedback and finalize \S~\ref{sec:mvp} and \S~\ref{sec:eval} of this document.

\subsection{Jan 31 2022: Receive White Papers from the Science Community}\label{ssec:time_wp}

{\bf June 1 2021:} Write a call for short scientific white papers that provide science-drive priorities and/or evaluate the relative performance of various photo-$z$ estimators, and advertise it broadly.
The format and contents of these white papers will reflect the minimum attributes and evaluation criteria that will (by then) be defined by this document; a preliminary draft is provided in \S~\ref{sec:wp}. \\
{\bf Jul, Sep, Nov, Jan 2021:} Hold four virtual one-hour ``face-to-face" LSST Photo-$z$ Forums that focus on allowing the project and the community discuss interim progress on their evaluation of photo-$z$ estimators. \\
{\bf Jan 31 2022:} Deadline for science white papers submitted to the DM team.

\subsection{April 30 2022: Select the Estimator(s)}\label{ssec:time_sel}

After review of the scientific white papers the Data Management team chooses one or more community-vetted estimators to be implemented.
The rational behind this selection will be added to this document, in \S~\ref{sec:choice}. 
After that final update, this document will cease to be {\it living} and will be archived as a static description of the process which culminated in the selection of a photo-$z$ estimator.

\subsection{2022 and Beyond: from Validation to Data Release}\label{ssec:time_opsdr}

During the Construction era, the Data Management team will implement and validate the selected photo-$z$ estimator(s) into the Data Release pipelines.
Preliminary guidelines for this process are collected in Appendix \ref{sec:imp} in order to inform the evaluation criteria (\S~\ref{sec:eval}).
Of particular interest to the science community might be Appendices \ref{ssec:imp_community} and \ref{ssec:imp_commissioning}, which provide some preliminary information about incorporating community efforts, and the potential role of commissioning data, in the photo-$z$ validation process.
Input from the community regarding the planned implementation and validation process is welcome at any time, and will be collected in Appendix \ref{sec:imp} and then propagated forward into the documents that eventually describe DM's ``as-built" implementation and validation software for {\tt Object} catalog photo-$z$ (which is beyond the scope of this document). 

During the Rubin Observatory Operations era, the Data Production team will produce and validate photo-$z$ for the {\tt Object} catalog using the software delivered by the Construction-era Data Management Team.
These photo-$z$ for the {\tt Object} catalog will be available at the time of all data releases, starting with DR1, along with any and all supporting materials such as documentation or spectral templates.
Rubin Operations may solicit and collect community feedback on the photo-$z$ performance, and return to any stage of this process to update the attributes, algorithms, implementation, and/or validation of {\tt Object} catalog photo-$z$ for future data releases.
Several alternative options for the generation of {\tt Object} catalog photo-$z$, such as photo-$z$ as in in-kind contribution from an international partnership or federating a user-generated catalog, are discussed in Appendix \ref{sec:opts}.


\clearpage
% % % % % % % % % % % % % % % % % % % % % % % % % % % % % %
\section{Proposed Minimum Attributes}\label{sec:mvp}

\textbf{These proposed minimum attributes will be refined via an iterative process between the Data Management team and the scientific community (\S~\ref{ssec:time_mvp}).} The supporting material in Appendices \ref{sec:dp} and \ref{sec:use} are also open for community input.

The following attributes are defined based on the guiding principle that the DMS-generated {\tt Object} photo-$z$ should, at minimum, {\it meet the basic science needs for communities which will not or cannot generate custom photo-$z$} and also DM's internal use-cases (Appendix \ref{sec:use}).

A {\it basic} science need would not include, for example, photo-$z$ of the quality required for major cosmological advances.
The development of such photo-$z$ algorithms is an active research topic within the Dark Energy Science Collaboration \citep{2018arXiv180901669T}, and a significant effort which the Rubin Observatory staff should not (and could not) attempt to replicate. 

{\bf Type of Estimator}\\
The {\tt Object} catalog photo-$z$ should be based on a template-fitting algorithm {\it and then also} a machine-learning algorithm, if multiple results can be stored (see Appendix \ref{ssec:dp_store}).
The main motivation for this is galaxies-related science (Appendix \ref{sssec:use_sci_gal}) and internal (Appendix \ref{ssec:use_dm}) use-cases, which would use the best-fit template to derive additional galaxy properties.

{\bf Output Format}\\
The {\tt Object} catalog photo-$z$ should include the posterior distribution function and a point estimate with an uncertainty, as well as reliable uncertainties and/or flags to help the novice user avoid mis-applying the results, or over-estimating their significance.
For template-fit photo-$z$ results, an identifier for the best-fit template should be provided, and those templates be made publicly accessible.
A list of the photo-$z$ outputs (and documentation) is put forth in Appendix \ref{ssec:dp_pz}.

{\bf Performance}\\
Based on the science use-cases in Appendix \ref{sec:use}, the {\tt Object} catalog photo-$z$ could have a point-estimate accuracy of $\sim10\%$ and still meet the basic science needs.
The photo-$z$ results should have a standard deviation in $z_{\rm true}-z_{\rm phot}$ of $\sigma_z < 0.05(1+z_{\rm phot})$, and a catastrophic outlier fraction of $f_{\rm outlier} < 10\%$, over a redshift range of $0.0 < z_{\rm phot} < 2.0$ for galaxies with $i<25$ mag galaxies.


\clearpage
% % % % % % % % % % % % % % % % % % % % % % % % % % % % % %
\section{Proposed Evaluation Criteria} \label{sec:eval}

\textbf{These proposed evaluation criteria will be refined via an iterative process between the Data Management team and the scientific community (\S~\ref{ssec:time_mvp}).} The supporting material in Appendices \ref{sec:imp}, \ref{sec:dp}, and \ref{sec:use} are also open for community input.

These criteria will be used to define the format and contents of community white papers evaluating the scientific potential of various photo-$z$ estimators, which will be solicited by Data Management as part of the estimator selection process (\S~\ref{ssec:time_wp}).

Criteria for evaluating the statistical performance of different photo-$z$ estimators during the selection process overlap with the validation tests discussed in Appendix \ref{ssec:imp_val}, especially the truth comparisons.
These criteria may also apply to community-generated catalogs proposing to be federated with the {\tt Objects} table (or be otherwise served at scale to the broader science community; Appendix \ref{ssec:opts_ugfed}).

{\bf Meeting the Defined Minimum Attributes}\\
The estimator should meet the minimum attributes defined in \S~\ref{sec:mvp}, and serve both the basic science needs of the science community and the internal use-cases of the Data Management team (Appendix \ref{sec:use}).
Performance beyond the basic needs, and the ability to provide higher-quality photo-$z$ results, should be considered favorably.
Note that the preliminary proposed performance (version 3 of this document; Sep 2020) is achievable for LSST data with existing photo-$z$ estimators, as demonstrated by, e.g., \citet{2018AJ....155....1G}, \citet{2020arXiv200103621S}.
Examples of journal articles that evaluate and compare the performance of photo-$z$ estimators are provided in \S~\ref{ssec:eval_eval}.

{\bf Scientific Utility and User Experience}\\
The selected estimator(s) should serve as wide a variety of science needs as possible.
Demonstrated success with other wide-field optical surveys, previous application to surveys that overlap with LSST (for results comparisons), and general community support or adoption of the estimator should all be considered.
The photo-$z$ estimator's estimators and output should be straightforward to understand and easy to access.
The selected estimator(s) should have detailed, publicly accessible documentation (e.g., a journal article, a website, a schema) to facilitate community use.
Desired photo-$z$ outputs, access methods, and documentation are discussed in Appendix \ref{ssec:dp_pz}.

{\bf Implementation, Validation, and Computational Requirements}\\
The implementation process (Appendix \ref{sec:imp}) should be straightforward, and the photo-$z$ estimator(s) must be able to be integrated with the DR pipelines. 
New algorithmic development is beyond the scope of DM, and the selected estimator should not need any expansion or maturation prior to installation in the DMS.
The selected estimator should take as inputs quantities that the LSST pipelines are planning to produce (Appendix \ref{ssec:dp_objvals}).
The necessary data -- from LSST or elsewhere -- for training, calibration, and validation must exist by the time of DR1 (e.g., galaxy catalogs with spectroscopic redshifts and LSST photometry; Appendix \ref{ssec:dp_calib}).
The computational needs of the selected estimator must fit within the available resources of the data facility.
%\textcolor{red}{The process of implementation and validation should require no more personnel hours than are available for the task.}


\subsection{Examples of Evaluating Photo-$z$ Estimator Performance}\label{ssec:eval_eval}

\begin{itemize}
\item \citet{2010A&A...523A..31H} tested 18 different photo-$z$ codes on the same sets of simulated and real data and found no significantly outstanding method.
\item \citet{2013ApJ...775...93D} tested 11 different photo-$z$ codes on the CANDLES data set ($U$-band through infrared photometry) and also find that no method stands out as the "best'', and that most of the photo-$z$ codes underestimate their redshift errors.
\item \citet{2014MNRAS.445.1482S} used the science verification data (200 square degrees of $grizY$ photometry to a depth of $i_{AB}=24$ magnitudes) of the Dark Energy Survey (DES) to evaluate several photometric redshift estimators. They found that the Trees for Photo-$z$ code (TPZ; \citet{2013ascl.soft04011C}) provided the most accurate results with the highest redshift resolution, and that template-fitting methods also performed well -- especially with priors -- but that in general there was no clear "winner.''
\item \citet{2018PASJ...70S...9T} provides a comparative analysis of several photo-$z$ estimators applied to their data set from the HSC Strategic Program. Their website\footnote{\url{https://hsc-release.mtk.nao.ac.jp/doc/index.php/photometric-redshifts/}} provides comparative analysis plots for each of them.
\item \citet{2020arXiv200103621S} statistically compare the posterior probability distribution functions produced by 12 photo-$z$ estimators for a mock data set that is representative of the LSST, identifying some biases and shortfalls in both the produced PDFs and the evaluation methods used to analyze them.
\end{itemize}


\clearpage
% % % % % % % % % % % % % % % % % % % % % % % % % % % % % %
\section{Proposed Call for Photo-$z$ White Papers} \label{sec:wp}

\textit{TBD: this draft proposed call for white papers will be revised after collecting community input on the minimum attributes and evaluation criteria.}

\textbf{Draft:} The Rubin Observatory LSST Data Management team solicits white papers that advocate for or against photo-$z$ estimators, using the evaluation criteria defined in \S~\ref{sec:eval}.
The DM team's goal is to select one or more photo-$z$ estimator(s) that will serve as broad a range of LSST science goals as possible. 
The length and format of the white papers are open, but they should focus on providing a clear and scientifically-justified prioritization of photo-$z$ estimators.
White papers may focus on many, a few, or just one estimator(s) or science goal(s), and may be qualitative (i.e., do not need to contain quantitative analyses). 

Note that because the timing and suitability of the data previews during commissioning is currently unclear (as of version 3 of this document; \S~\ref{ssec:imp_commissioning}), the science community's contributions to the photo-$z$ evaluation and selection process is likely to rely on simulated LSST data, or data from other facilities.

\clearpage
% % % % % % % % % % % % % % % % % % % % % % % % % % % % % %
\section{Rationale for DM's Selected Photo-$z$ Estimator(s)} \label{sec:choice}

\textit{TBD: this section will describe which community-vetted photo-$z$ estimator(s) DM has selected to implement, and why.}



\clearpage
% % % % % % % % % % % % % % % % % % % % % % % % % % % % % %
\bibliography{lsst,refs_ads,local}


\clearpage
% % % % % % % % % % % % % % % % % % % % % % % % % % % % % %
\appendix 

%% % % % % % % % % % % % % %
%\subsection{Mario's Original Roadmap to Photo-$z$}\label{ssec:dmcalc_mario}
%This option has served as the basis for our discussions with the science community since 2016, and is based on comments by Mario Juri\'{c} on the Jira thread DM-6367.
%The process for the definition of the Data Release (DR) photo-$z$ data product is:
%\begin{enumerate}[noitemsep,topsep=-10pt]
%\item DM Project Science (acting on behalf of overall Project Science) consults with the community (represented by the relevant collaboration -- in this case, the DESC) for proposed options regarding the most appropriate algorithm and format for the results. Iteration between DM and the science community will be necessary in order to converge to a scientifically acceptable {\it and} implementable solution (e.g., if DESC recommends an algorithm that does not meet DM's computational, storage, or budget limitations, or if DESC recommends an algorithm that the other SC have concerns with).
%\item Following that (iterative) consultation, DM will recommend (and the Project will select) the photo-$z$ algorithm and the data product format to be incorporated into DR Processing.
%\item DM will implement the selected algorithm, using whatever they can transfer over from DESC (or other) work, but DM's requirements may be higher than DESC's (e.g., the algorithm will need to run reliably in LSST's production environment).
%\item DM will implement anything needed to integrate, verify, validate, and run QA on the photo-$z$ data product in data releases (and everything leading up to the annual releases; e.g., commissioning), because photo-$z$ are ultimately a Project deliverable.
%\end{enumerate}

\clearpage
% % % % % % % % % % % % % % % % % % % % % % % % % % % % % %
\section{Appendix: Additional Options for {\tt Object} Catalog Photo-$z$}\label{sec:opts}

\subsection{Photo-$z$ as an In-Kind Contribution}\label{ssec:opts_inkind}

The process outlined by this document for Data Management to select, implement, and validate a photo-$z$ estimator into the Data Release pipeline would also apply to any international team providing photo-$z$ for the {\tt Object} catalog as an in-kind contribution in exchange for data rights. 
This team should expect to work closely with Data Management on each step of the proposed process in \S~\ref{sec:time}, including the solicitation of community input, selecting the estimator, the process of implementation and validation, and then maintaining and evolving the photo-$z$ estimator during Rubin Operations (\S~\ref{ssec:time_opsdr}).

Any in-kind contribution team providing photo-$z$ estimates for the {\tt Object} catalog should should also review the summary of related data products that they would be expected to assemble and maintain (Appendix \ref{sec:dp}), the example scientific use-cases (Appendix \ref{sec:use}), and the relevant documentation that pertains to the photo-$z$ (Appendix \ref{sec:docs}).

\subsection{Federation of a User-Generated Photo-$z$ Catalog}\label{ssec:opts_ugfed}

As described in \S~\ref{sec:mvp}, the DMS as delivered to the Operations Project will provide a minimal scientific capability with respect to {\tt Object} catalog photo-$z$.
If that is rendered obsolete by community efforts, and a superior user-generated photo-$z$ catalog is validated (\S~\ref{ssec:imp_val}), {\it and} its creators are willing to share, then this data product should be ingested ("federated") and served with the {\tt Object} catalog.

To facilitate this option might require providing the community team(s) with access to data release previews\footnote{Data release previews are envisioned to be some small, e.g., $\sim10\%$, amount of a data release made available early, e.g., weeks to months, before the full release to enable community feedback and preparations prior to the full release.} so that they may train and calibrate their photo-$z$ estimator, and minimize the time between data release and federation of a new photo-$z$ catalog to the {\tt Objects} table.

Facilitating multiple teams sharing their photo-$z$ catalogs might involve hosting a "photo-$z$ server" within the Rubin Science Platform.
For example, the Dark Energy Survey's Science Portal is an infrastructure for organizing input catalogs, installing photo-$z$ estimators, training and running them, and evaluating their output\footnote{A series of YouTube tutorials about the DES Science Portal are available at \url{https://www.youtube.com/playlist?list=PLGFEWqwqBauBIYa8H6KnZ4d-5ytM59vG2}.}, as described by \citet{2018A&C....25...58G}).


\subsection{The Science Impacts of a Data Release Without {\tt Object} Photo-$z$}\label{ssec:opts_none}

The option to federate a community-generated photo-$z$ catalog leads to this question: {\it If there will likely be a superior community photo-$z$ anyway, should Data Management avoid installing a photo-$z$ estimator in the DMS, and instead simply wait for the community to generate a catalog?}
There are several significant risks and drawbacks to this option.
\vspace{-15pt}
\begin{itemize}
\item An {\tt Object} table without photo-$z$ at the time of data release is a problem for brokers, unless alerts are instead (or additionally) associated to an older DR's {\tt Object} catalog that has photo-$z$. Brokers require host-galaxy photo-$z$ to optimally classify and prioritize transients for follow-up, and plan to obtain this information via each alert's associations with nearby {\tt Objects} from {\it the most recent DR}.
\item The {\tt Object} photo-$z$ might be tailored to the specific science case of the community team and might not serve the broader science use-cases.
\item There is no initiative or reward for the community team which has generated the photo-$z$, except perhaps citations to their photo-$z$ catalog.
\item There would be no {\tt Object} photo-$z$ if no community team generates and donates a catalog, which is a risk for the science use-cases described in Appendix \ref{ssec:use_sci}
% \item This option does not satisfy a literal interpretation of the requirement\dmreq{0046} that the {\it "DMS shall compute"} photo-$z$.
\end{itemize}


\clearpage
% % % % % % % % % % % % % % % % % % % % % % % % % % % % % %
\section{Appendix: The Implementation and Validation Process}\label{sec:imp}

\textbf{This preliminary collection of guidelines for the implementation process and validation tests is open to community input during the roadmap phase in \S~\ref{ssec:time_mvp}.}

The DM team will be responsible for implementing the estimator(s) and validation tests into the DR pipeline along with any necessary supporting data.
It is anticipated that this process might deviate from these preliminary guidelines because some details of implementation and validation will be particular to the selected photo-z estimator(s).
The final, ``as-built" implementation and validation process would be documented elsewhere.

The proposed set of basic steps for implementing a photo-$z$ estimator are:
\vspace{-15pt}
\begin{itemize}
\item prepare the LSST data inputs to the photo-$z$ estimators (Appendix \ref{ssec:dp_objvals})
\item prepare the training and calibration data (Appendix \ref{ssec:dp_calib})
\item install the needed codes (a technical aspect to be described elsewhere)
\item run the estimator (on its own and/or embedded in the data release pipeline)
\item validate the photo-$z$ estimator outputs (below, \ref{ssec:imp_val})
\item store the results in the {\tt Object} catalog (Appendices \ref{ssec:dp_pz} and \ref{ssec:dp_store})
\item ensure output schema and access methods are documented (Appendix \ref{ssec:dp_pz})
\end{itemize}

\subsection{Proposed Validation Tests}\label{ssec:imp_val}

Validation tests and quality assessment diagnostics will be necessary to ensure the results meet performance expectations (\S~\ref{sec:mvp}).
Below is a compilation of potential validation tests, some of which overlap with the evaluation criteria described above (especially the truth comparisons). 
These lists are based in part on a brainstorming session during the LSST Project and Community Workshop's session on Photometric Redshifts on Aug 14 2019\footnote{E.g., slide 14 of \url{https://docs.google.com/presentation/d/1GEahvDQXIjSL4lLVjDlZHV5zpXLhGQfHwVtucs72Ajg/edit?usp=sharing}}.

Journal articles that demonstrate validation processes for photo-$z$ from multi-band wide-area surveys include {\it "DES science portal: Computing photometric redshifts"} \citep{2018A&C....25...58G}; {\it "Photometric redshifts for Hyper Suprime-Cam Subaru Strategic Program Data Release 1"} \citep{2018PASJ...70S...9T}; and {\it "On the realistic validation of photometric redshifts"} \citep{2017MNRAS.468.4323B}. Furthermore, DESC is currently investigating methods of validating accuracy of probability distributions from a photo-$z$ estimator (e.g., \citealt{2020arXiv200103621S}).

{\bf Truth Comparisons}\\
Catalogs of true redshifts can be obtained by withholding some fraction of the training set or by cross-matching to external spectroscopic catalogs. Validation metrics should include at least those used to define the minimum performance of the LSST photo-$z$ for basic science needs (\S~\ref{sec:mvp}). A list of potential metrics might include the following, and targets or limits on these metric values might apply to subsets in magnitude, color, or redshift: 
\vspace{-15pt}
\begin{itemize}
\item plots of $z_{\rm true}$ {\it vs.} $z_{\rm phot}$ for visual inspection
\item standard deviation and bias in $\Delta z = (z_{\rm true}-z_{\rm phot})/(1+z_{\rm true})$
\item fraction of (catastrophic) outliers, e.g., $\Delta z > 3\sigma$ or $>0.06$ ($>1$)
\item quantile-quantile (q-q) plots evaluate the shape of $P(z)$
\item probability integrated transform, $PIT(z_{\rm phot}) = \int_{0}^{z_{\rm phot}} P(z)\,dz$, calculated for all test galaxies, should be flat for well-estimated $P(z)$ \citep{2016arXiv160808016P}
\item the continuous ranking probability score, CRPS, should be a lower value for better estimated $P(z)$, as described in \citep{2016arXiv160808016P}
\item redshift confidence for point estimates, $C(z_{\rm phot}) = \int_{z_{\rm phot}-0.03}^{z_{\rm phot}+0.03} P(z)\,dz $
\item loss and risk parameters that characterize the photo-$z$ estimates:\\
$L(\Delta z) = 1 - \left(1+ \left(\frac{\Delta z}{\gamma} \right)^2 \right)^{-1}$, 
where $\gamma$ is a characteristic threshold (e.g., $0.15$), and:\\
$R(z_{\rm phot}) = \int P(z)\,L(z_{\rm phot},z)\,dz$, as described in \citet{2018PASJ...70S...9T}
\item the conditional density estimation (CDE) loss as described in Section 4.2 of \citet{2020arXiv200103621S}. (Note that the CDE does not strictly require the true posterior be known.)
\end{itemize}

{\bf Sanity Checks}\\
These are tests that can be done using all LSST {\tt Objects} and do not require a truth catalog.
\vspace{-15pt}
\begin{itemize}
\item the uncertainty in $z_{\rm phot}$ should correlate with photometric error
\item the star/galaxy flag parameter should agree with photo-$z$ (stars have $z_{\rm phot}=0$)
\item evolution of $N(z)$ to higher-$z$ for samples with fainter magnitudes, redder colors
\end{itemize}

{\bf Scientific Applications}\\
The following might not be included in the formal validation process, but represent analyses that the broader scientific community may want to do to inform their use of the LSST photo-$z$.
\vspace{-15pt}
\begin{itemize}
\item assessing the performance of point estimates in broker photometric classification algorithms
\item evaluating the absolute magnitude distribution of Type Ia supernovae once the distance derived from the photo-$z$ point estimates are applied
\item evaluating the distributions of derived physical parameters for galaxies using the $P(z)$
\item checking whether high SFR galaxies (and maybe other sensitive populations) have reasonable $P(z)$
\item galaxy cluster membership identification
\item tomographic bin analysis, as in \citep{2019MNRAS.482.2807C}
\end{itemize}


\subsection{Community Contributions to Photo-$z$ Validation}\label{ssec:imp_community}

{\bf Community input regarding what is needed to enable participation in the validation process is welcome during the roadmap phase in \S~\ref{ssec:time_mvp}.} Additional details will be provided by the Data Management team in the future, as the possible modes of community science validation become better understood.

It is desirable that the photo-$z$ implementation and validation process include contributions from the scientific community.
There are several aspects to consider when enabling such contributions.
For example, to enable such contributions, the community will need access to LSST-like data products that are generated from, e.g., HyperSuprimeCam survey data or commissioning data from the Rubin LSST Science Camera (but see \S~\ref{ssec:imp_commissioning}).
The community would also benefit from having this access through the Rubin Science Platform, to facilitate the communal development and sharing of codes related to validation.
For example, a system like the Dark Energy Survey's Science Portal, though which users may train, run, and evaluate photo-$z$ estimators \citet{2018A&C....25...58G}.

\subsection{The Role of Commissioning Data}\label{ssec:imp_commissioning}

{\bf Community input regarding what type of commissioning data would enable photo-$z$ scientific validation is welcome during the roadmap phase in \S~\ref{ssec:time_mvp}.} Additional details about commissioning data will be provided to the community in the future, as the situation becomes clearer.

As of the release of version 3 of this document in Sep 2020, the timing and plans for LSST commissioning were unclear; preliminary plans for the data previews are described below.
It does appear likely that the science community's white papers evaluating various photo-$z$ estimator(s) will have to be completed with simulated LSST data, or data from other facilities, because that commissioning data will not be available prior to the set time to select (a) photo-$z$ estimator(s).

Commissioning data is likely to be used by DM for the photo-$z$ implementation and validation process, and the goal will be to make this available to the community if possible, so that they can contribute to the validation process.
However, it should not be {\it assumed} that any release of data products based on commissioning surveys will include photo-$z$ estimates: the release of any photo-$z$ catalogs prior to DR1 remains at the discretion of the Data Management team.

{\bf Data Preview 1 (DP1):} The first phase of commissioning with ComCam might include, e.g., tests of the scheduler; tests of image quality, depth, astrometry, and photometry; and a 20-year depth test to stack images over a range of conditions \citedsp{LSE-79}.
DP1 might be useful as preliminary test data during the implementation of the photo-$z$ estimators, and help to test the formats of the input and output data, create visualizations for the validation codes, etc.

{\bf Data Preview 2 (DP2):} The second phase of commissioning with the Science Camera might include, e.g., wide area surveys to the full 10-year depth in addition to a 20-year depth stack and tests of image quality, as in DP1 \citedsp{LSE-79}.
DP2 will provide the data set for the photo-$z$ validation process described in \S~\ref{ssec:imp_val}, and also some of the needed training data for LSST photo-$z$ described in Appendix \ref{ssec:dp_calib}.



\clearpage
% % % % % % % % % % % % % % % % % % % % % % % % % % % % % %
\section{Appendix: Data Products Related to LSST Photo-$z$}\label{sec:dp}

\textbf{The contents of the following sections will be refined over time to eventually describe the data products input and output from the selected photo-$z$ estimator(s), and contributions from the scientific community are solicited for each of the following topics.}

This section is a preliminary collection of details related to generating and serving LSST photo-$z$, such as the LSST data products that will be needed as input to photo-$z$ estimators (Appendix \ref{ssec:dp_objvals}), the required training and calibration data from LSST and external sources (Appendix \ref{ssec:dp_calib}), the outputs' format, access methods, and documentation (Appendix \ref{ssec:dp_pz}), and options for compressing photo-$z$ results to potentially store the results of multiple estimators (Appendix \ref{ssec:dp_store}).


\subsection{Inputs to Photo-$z$ Estimators}\label{ssec:dp_objvals}

It is important to ensure that all measured quantities needed by photometric redshift estimators are going to be computed and included in the {\tt Object} table. 

Aside from the fluxes and/or apparent magnitudes and errors for each Rubin Observatory filter, which will be provided in the {\tt Object} catalog, the color properties in the {\tt Object} table might be used for photo-$z$. {\it "Colors of the object in 'standard seeing' (for example, the third quartile expected survey seeing in the i band, $\sim$0.9 arcsec) will be measured. These colors are guaranteed to be seeing-insensitive, suitable for estimation of photometric redshifts"} \citedsp{LSE-163}. In the {\tt Object} table the relevant elements are:
\vspace{-15pt}
\begin{itemize}
\item \texttt{stdColor (float[5])} = {\it 'standard color', color of the object measured in 'standard seeing', suitable for photo-$z$}
\item \texttt{stdColorErr (float[5])} = {\it uncertainty on \texttt{stdColor}}
\end{itemize}

Additionally, measured quantities such as the galaxy size, shape, radial profile, 'clumpiness', or surface brightness; the DCR correction (or residual); or a parameter that represents the clustering density within some radius (e.g., 2 Mpc) might all be useful (e.g., as priors) for photo-$z$ estimators. The effective transmission function ($\phi$; Eq. 5 in \citeds{LPM-17}), which will be provided for all {\tt Sources} either in the catalog or as a link, is another useful quantity for photo-$z$ estimators.


\subsection{Training and Calibration Data}\label{ssec:dp_calib}

Regardless of whether the {\tt Object} catalog photo-$z$ are generated by the DMS or an ingested community catalog, some training and calibration data from LSST will be needed.

{\bf Spectroscopic Redshifts}\\
Deep multi-band LSST photometry for spectroscopic fields like COSMOS, and/or WFD-depth LSST photometry that overlaps multiplexed spectroscopic surveys like DESI/4MOST, which is obtained either during commissioning or early in Operations year 1, will likely be necessary to produce photo-$z$ for DR1.

{\bf Wide Area Imaging}\\
Some photo-$z$ methods have requirements other than spec-$z$ fields: e.g., \citet{2019MNRAS.483.2801S} use clustering information to obtain photo-$z$ and this requires wider, shallower field coverage and not a single deep pointing like a spec-$z$ field would have. 
This wider area would also serve to reduce cosmic variance in the training set ($\sim$100 square degrees would serve to average out the variance).

{\bf Data Previews}\\
For the community to participate in the training or calibration of a photo-$z$ estimator prior to a data release, it will probably be necessary to release a small ($\sim10\%$) but representative "DR preview" in advance of each data release during Operations.


\subsection{Output Schema, Access Methods, and Documentation}\label{ssec:dp_pz}

This section gathers details regarding what the photo-$z$ outputs should be, how they should be accessed, and how they should be documented.

{\bf Output Schema}\\
The format of the photo-$z$ output is one of the minimum attributes that must still be defined (\S~\ref{sec:mvp}). 
The photo-$z$ outputs must provide all the necessary inputs to the validation tests, to be defined in \S~\ref{ssec:imp_val}.
The currently proposed {\tt Object} catalog table elements related to photo-$z$ are defined in \citeds{LSE-163} and provided in \S~\ref{ssec:docs_dpdd}, and summarized here:
\vspace{-15pt}
\begin{itemize}
\item posterior probability distribution function (likelihood over redshift)
\item a single point estimate with an uncertainty
\item quantities related to the posterior (e.g., mode, mean, skewnewss)
\item flags (e.g., potential catastrophic outlier, failure mode, consistent with z=0)
\end{itemize}

{\bf Access Methods}\\
The user experience is one of the proposed selection criteria for the LSST photo-$z$ estimator (\S~\ref{sec:eval}). 
Some examples of publicly released photo-$z$ catalogs which were prepared with a user experience that might be desirable for the LSST photo-$z$ include the Dark Energy Survey's Science Portal to serve photometric redshifts \cite{2018A&C....25...58G} and the Hyper SuprimeCam Subaru Strategic Program \cite{2018PASJ...70S...9T}\footnote{\url{https://hsc-release.mtk.nao.ac.jp/doc/index.php/photometric-redshifts/}}.

If the LSST photo-$z$ are not made available in either the {\tt Objects} table or in a federated or joinable catalog -- for example in the case where a community-generated photo-$z$ catalog is replacing the DMS-generated catalog (Appendix \ref{ssec:opts_ugfed}) -- and are instead made available via, e.g., a "photo-$z$ server" (as in \cite{2018A&C....25...58G}), then at least the {\tt Object} catalog ID of the most recent data release should be a queryable parameter.

If the results of multiple estimators are generated, compressed, and stored in the {\tt Objects} table, then decompression should be straightforward for the user (Appendix \ref{ssec:dp_store}).

{\bf Documentation}\\
Appropriate types of documentation might include published journal articles, GitHub repositories, websites, or other online documentation resources (e.g., \url{https://readthedocs.org/}). Whatever the format, the documentation contents should include: 
\vspace{-15pt}
\begin{itemize}
\item general description of the estimator
\item adaptations made to ingest LSST data (compared to past applications)
\item an analysis of the training, calibration, and validation processes
\item a full list of all inputs and outputs (i.e., a schema browser)
\end{itemize}

\subsection{Storage and Compression}\label{ssec:dp_store}

Regardless of whether the {\tt Object} catalog photo-$z$ are generated by the DMS or an ingested community catalog, the stored values are subject to the storage space allotted in the {\tt Objects} table as described in \S~\ref{ssec:docs_dpdd}.
However, both the posteriors and point estimates from several different photo-$z$ estimators could be compressed and stored in this allotted space.
Furthermore, given the variety of use-cases and the fact that different photo-z estimators produce different results \citep{2020arXiv200103621S}, the option to compute, compress, and store estimates from multiple estimators in the $2\times95$ float might be scientifically desirable.

Efficient $P(z)$ compression algorithms are in development, such as \citet{2014MNRAS.441.3550C} and \citet{2018AJ....156...35M}.
\citet{2014MNRAS.441.3550C} present an algorithm for sparse representation, for which {\it "an entire PDF can be stored by using a 4-byte integer per basis function''} and {\it "only ten to twenty points per galaxy are sufficient to reconstruct both the individual PDFs and the ensemble redshift distribution, $N(z)$, to an accuracy of 99.9\% when compared to the one built using the original PDFs computed with a resolution of $\delta z = 0.01$, reducing the required storage of two hundred original values by a factor of ten to twenty.''} 
\citet{2018AJ....156...35M} presents a {\tt Python} package for compressing one-dimensional posterior distribution functions (PDFs), demonstrates its performance on several types of photo-$z$ PDFs, and provides a set of recommendations for best practices which should be consulted when DM is making decisions on the DR photo-$z$ data products.

However, compression (and decompression by users) will require extra computational resources, which should be estimated and considered, and decompression must be fast and easy for users.


\clearpage
% % % % % % % % % % % % % % % % % % % % % % % % % % % % % %
\section{Appendix: Example Use-Cases for LSST Photo-$z$} \label{sec:use}

This section contains an incomplete, non-exhaustive summary of a variety of internal and scientific use-cases for the LSST {\tt Object} catalog photo-$z$.
These use-cases inform the minimum attributes and selection criteria proposed in \S~\ref{sec:mvp} and \ref{sec:eval}. 

Some of the following information on use-cases was collected from participants of the LSST Project and Community Workshop's session on Photometric Redshifts on Aug 14 2019\footnote{Thanks to Sam Schmidt, Chris Morrison, Sugata Kaviraj, Gautham Narayan, Lauren Corlies, Travis Rector, Tina Peters, Alex Malz, Dara Norman, Stephen Smartt, and other participants from the science community.}.

\subsection{Internal DMS Use-Cases}\label{ssec:use_dm}

DM's galaxy photometry outputs are being developed with the goal of feeding photometric redshift estimators, so the computation of photometric redshifts is likely to be a part of the science validation process for LSST photometry. 
Unlike stars, color-color and color-magnitude diagrams for galaxies do not have sufficient structure to reveal issues with the photometry.
While other photometric validation techniques will also be useful (such as evaluating the width of galaxy cluster red-sequences) they may only apply to {\it some} galaxies, whereas {\it all} galaxies have a redshift. 

The internal use-case of scientifically validating the galaxy photometry outputs is likely to require a simple photo-$z$ estimator which fits SED templates, since the goal is to evaluate whether the photometric outputs match the colors of real galaxies.
Whether such a simple SED-fit photo-$z$ could also serve the scientific use-cases is undetermined, because the photometric validation process is not yet defined or written.
The internal use-case described in \S~\ref{ssec:docs_oss} -- of needing photo-$z$ in order to assess catalog completeness for low- and high-redshift {\tt Objects} -- is also likely to be served by a simple SED-fit photo-$z$ estimate.

As a side note, although the DMS will assign fiducial spectral energy distributions (SEDs) to {\tt Objects} in order to apply sub-band wavelength-dependent photometric calibration and PSF modeling, computing photo-$z$ is not planned to be a part of this process.
Furthermore, the SED templates used will likely be simpler (e.g., step-function or slope) than would be needed for deriving photo-$z$.

\subsection{Scientific Use-Cases}\label{ssec:use_sci}

A variety of potential scientific applications for the {\tt Object} photo-$z$ are discussed in turn. 
These scientific use-cases should be used to inform the minimum attributes and selection criteria proposed in \S~\ref{sec:mvp} and \ref{sec:eval}.
A summary of the commonalities between science use-cases for photo-$z$ is provided in \S~\ref{sssec:use_sci_sum}.

\subsubsection{Dark Energy}\label{sssec:use_sci_de}
Extragalactic astrophysics such as weak lensing, baryon acoustic oscillations, and Type Ia supernova cosmology are all main science drivers for the LSST, and all require catalogs of galaxies with photometric redshifts.
The photo-$z$ estimators for precision cosmology will be custom-tailored to these particular science goals, and the photo-$z$ results are subject to established science requirements for dark energy cosmology \citep{2018arXiv180901669T}.
For example, weak lensing and large scale structure require ensemble measurements of $N(z)$ and thus require a full posterior PDF, whereas point-estimate photo-$z$ for individual {\tt Objects} are required for Type Ia supernova host galaxies and the identification of strong lensing candidates and galaxy cluster members. 
The Dark Energy Science Collaboration (DESC) is developing specialized photo-$z$ pipelines for these science goals (which {\it could} serve to generate photo-$z$ for the {\tt Object} catalog, as discussed in Appendix \ref{ssec:opts_ugfed}).

\subsubsection{Time Domain}\label{sssec:use_sci_td}
The Transients and Variable Stars Science Collaboration reported that they would use LSST-provided {\tt Object} photo-$z$ to identify and/or characterize extragalactic transient host galaxies.
Alert packets provide {\tt Object} IDs for the three nearest stars and three nearest galaxies in the most recent data release.
Alert stream brokers intend to query the {\tt Objects} catalog in real time to obtain host photo-$z$ because photometric classification for transient light curves is {\it significantly} aided by redshift estimates.
The {\tt Object} catalog's photo-$z$ will also be used to identify and prioritize the potential host galaxies of gravitational wave events for imaging searches of the optical counterpart.
% In fact, if nearest neighbors characteristics (Reff) and photo-z could be in the packet, that would enable faster classifications

\subsubsection{Galaxies}\label{sssec:use_sci_gal}
The Galaxies Science Collaboration reported that they would use LSST-provided {\tt Object} photo-$z$, and that their science goals require that photo-$z$ be accurate enough ($<10\%$) to derive intrinsic galaxy properties like mass and star formation rate (SFR).
They also indicated that posteriors delivered as $P(z,M)$ and/or with rest-frame apparent magnitudes would be useful to their science goals.
This indicates that the results of a template-fitting photo-$z$ estimator might be more relevant to Galaxies studies than machine-learning estimates (especially if the SED templates are associated with intrinsic galaxy properties like mass, metallicity, or star formation rate).
The {\tt Object} photo-$z$ might also be used to assist with star-galaxy separation, to enable population studies, to estimate environmental (clustering) parameters, and/or to choose instrument configurations for spectroscopic follow-up (i.e., the expected location of emission lines).

\subsubsection{Active Galactic Nuclei}\label{sssec:use_sci_agn}
It is currently unclear how useful the {\tt Object} photo-$z$ will be for the AGN community because there is no special deblending planned for the DMS to produce galaxy photometry which is free of AGN emission.
The AGN contribution to the DR CoAdd image stacks, and thus the {\tt Object} catalog photometry, will be an average flux over the LSST survey images.
Photometric redshift codes will either have to be able to recognize and deal with AGN contamination, or the photo-$z$ estimates for AGN host galaxies will be impacted.
Potential AGN contamination could be identified by identifying {\tt DIAObjects} in the nuclear region, but quantifying and removing that AGN flux from the galaxy photometry and recalculating photo-$z$ remain a user-generated data product.

\subsubsection{Clustering}\label{sssec:use_sci_clust}
Photometric redshifts would likely be used by individuals studying large scale structure and galaxy clustering -- for example, as a way to make an initial selection of cluster members.

\subsubsection{Stars, Milky Way, and Local Volume}\label{sssec:use_sci_smwlv}
LSST-provided {\tt Object} photo-$z$ could be used to reject compact extragalactic objects from stellar samples for population studies and/or spectroscopic follow-up campaigns.

\subsubsection{Education and Public Outreach}\label{sssec:use_sci_epo}
The question {\it "how far away is it?"} is common to many EPO initiatives and the {\tt Object} catalog photo-$z$ will be used when preparing information for the public.
EPO might also use photo-$z$ for, e.g., generating 3D graphics that visualize large volumes, or educational programs on the Hubble constant.
For EPO purposes, high precision is not as important as outlier reduction for photo-$z$.

%\subsubsection{Current Surveys}
%{\bf Placeholder to add citations to the science enabled by release of photo-$z$ catalogs for recent wide-area surveys (e.g., SDSS, HSC, DES?).}

\subsubsection{Science Use-Cases Summary}\label{sssec:use_sci_sum}
Aside from the specialized use-cases related to dark energy cosmology, which will be served by customized photo-$z$ estimators developed within DESC, most other scientific scenarios use the {\tt Object} photo-$z$ as point estimates of distance in order to subset the data and identifying targets of interest for follow-up, and/or infer intrinsic galaxy properties.

\subsection{Considerations for Maximizing Early Science}\label{ssec:use_LOY1}

To maximize early science capabilities, estimators that will return the most accurate photo-$z$ as early in the survey as possible could be prioritized.
In the first year of LSST, it might be simpler to use a template-fitting photo-$z$ estimator and avoid potential issues related to computation resources and/or the need to train a machine learning model.
Additionally, the large spectroscopic training sets needed for ML photo-$z$ estimators are more likely to exist by 2030 than at 2020.
However, if a machine learning estimator is applied for LSST DR 1 and 2, it should be a community-accepted estimator with demonstrated success in other surveys, preferably surveys that overlap the LSST volume, as this will facilitate the characterization and validation of the LSST photo-$z$.


\clearpage
% % % % % % % % % % % % % % % % % % % % % % % % % % % % % %
\section{Appendix: LSST Documentation Review}\label{sec:docs}

This section contains a review of all appearances of the terms "photometric redshift", "redshift", or "photo-$z$" in the LSST documentation.
The purpose of this review is to clarify the scope and expected deliverable quality of the {\tt Object} catalog photo-$z$, and any internal use-cases (as discussed in further detail in Appendix \ref{ssec:use_dm}).
% Note that none of these terms appear in the LSST System Requirements \citedsp{LSE-29}.

\subsection{Science Requirements Document}\label{ssec:docs_srd}

One of the main science drivers of the LSST design is a significant advance in constraining the models of dark energy cosmology. 
Section 2.1 of \citeds{LPM-17} describes the statistical accuracy of photo-$z$ estimates for $i<25$, $0.3<z_{\rm phot}<3.0$ galaxies which are required for the cosmological probes: root-mean-square error $<0.02(1+z_{\rm phot})$, bias $<0.003$, and fraction of catastrophic outliers $<10\%$.
The SRD specifies that these target statistical values {\it "are the primary drivers for the photometric depth of the main LSST survey."} 
In other words, the LSST 10-year photometry must enable a state-of-the-art photometric redshift estimator to achieve these targets -- they do not apply to the general-use {\tt Object} catalog photo-$z$ which are the topic of this document.

\subsection{Observatory System Specifications}\label{ssec:docs_oss}

There is a requirement\ossreq{0164} that {\it "the object catalog completeness"} shall be determined by the DMS for {\it "a variety of astrophysical objects"}, which includes {\it "small galaxies on both the red- and blue-sequence at a range of redshifts, and supernovae at a range of redshifts"} \citedsp{LSE-30}. 
For the DMS to meet this requirement and determine the object catalog completeness for these classes of objects, it requires redshift estimates for {\tt Objects}.
Although spectroscopic redshifts could be obtained and used for this purpose, that would require observing time with non-LSST facilities, and it is instead more feasible that photometric redshifts would be used to meet this requirement.

\subsection{Data Management System Requirements}\label{ssec:docs_dmsr}

% JIRA ticket DM-6367
There is a requirement\dmreq{0046} on the Data Management System (DMS) which states that {\it "The DMS shall compute a photometric redshift for all detected Objects"} \citedsp{LSE-61}. 

No discussion or details are provided regarding how or when the {\tt Object} photo-$z$ are to be calculated, validated, or served, or whether it might be equivalent to serve photo-$z$ computed by a third party (i.e., to federate a user-generated photo-$z$ catalog).
It is the current role of this document to evaluate the options for fulfilling this requirement and initiate an LSST Change Request to clarify the computation of {\tt Object} photo-$z$.

\subsection{Data Products Definitions Document}\label{ssec:docs_dpdd}

The LSST Data Products Definitions Document (DPDD) \citedsp{LSE-163} defines the format of the {\tt Object} catalog's table columns which could store the results of photometric redshift estimates, regardless of how they're generated. 
The following is from Table 5 of the DPDD:
\vspace{-15pt}
\begin{itemize}
\item \texttt{photoZ (float[2x95])} = photometric redshift likelihood samples -- pairs of redshift and likelihood ($z,\log{L}$) -- computed using a to-be-determined published and widely accepted estimator at the time of LSST Commissioning
\item \texttt{photoZ\_pest (float[10])} = point estimates for the photometric redshift provided in {\tt photoZ}
\end{itemize}

The exact point estimate quantities stored in the \texttt{photoZ\_pest} are to-be-determined, {\it "but likely candidates are the mode, mean, standard deviation, skewness, kurtosis, and 1\%, 5\%, 25\%, 50\%, 75\%, and 99\% points from cumulative distribution"} \citedsp{LSE-163}. 

\subsection{Data Management Science Pipelines Design}\label{ssec:docs_ldm151}

This document clarifies that the photo-$z$ estimator would not be developed by LSST DM, but that DM would be responsible for implementing the code to run on the entire {\tt Objects} catalog and validating the results: \\
{\it "In addition to data products produced by DM, a data release production also includes official
products (essentially additional Object table columns) produced by the community. These
include photometric redshifts and dust reddening maps. While DM's mandate does not extend
to developing algorithms or code for these quantities, its responsibilities may include validation
and running user code at scale"} \citedsp{LDM-151}.


\end{document}
